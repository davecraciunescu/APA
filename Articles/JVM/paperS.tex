% This template has been tested with LLNCS DOCUMENT CLASS -- version 2.20 (10-Mar-2018)

% !TeX spellcheck = en-US
% !TeX encoding = utf8
% !TeX program = pdflatex
% !BIB program = bibtex
% -*- coding:utf-8 mod:LaTeX -*-

% "a4paper" enables:
%  - easy print out on DIN A4 paper size
%
% One can configure a4 vs. letter in the LaTeX installation. So it is configuration dependend, what the paper size will be.
% This option  present, because the current word template offered by Springer is DIN A4.
% We accept that DIN A4 cause WTFs at persons not used to A4 in USA.

%/ "runningheads" enables:
%  - page number on page 2 onwards
%  - title/authors on even/odd pages
% This is good for other readers to enable proper archiving among other papers and pointing to
% content. Even if the title page states the title, when printed and stored in a folder, when
% blindly opening the folder, one could hit not the title page, but an arbitrary page. Therefore,
% it is good to have title printed on the pages, too.
%
% It is enabled by default as the springer template as of 2018/03/10 uses this as default

% German documents: pass ngerman as class option
% \documentclass[ngerman,runningheads,a4paper]{llncs}[2018/03/10]
% English documents: pass english as class option
\documentclass[english,runningheads,a4paper]{llncs}[2018/03/10]

%% If you need packages for other papers,
%% START COPYING HERE

% Set English as language and allow to write hyphenated"=words
%
% In case you write German, switch the parameters, so that the command becomes
%\usepackage[english,main=ngerman]{babel}
%
% Even though `american`, `english` and `USenglish` are synonyms for babel package (according to https://tex.stackexchange.com/questions/12775/babel-english-american-usenglish), the llncs document class is prepared to avoid the overriding of certain names (such as "Abstract." -> "Abstract" or "Fig." -> "Figure") when using `english`, but not when using the other 2.
% english has to go last to set it as default language
\usepackage[ngerman,main=english]{babel}
%
% Hint by http://tex.stackexchange.com/a/321066/9075 -> enable "= as dashes
\addto\extrasenglish{\languageshorthands{ngerman}\useshorthands{"}}
%
% Fix by https://tex.stackexchange.com/a/441701/9075
\usepackage{regexpatch}
\makeatletter
\edef\switcht@albion{%
  \relax\unexpanded\expandafter{\switcht@albion}%
}
\xpatchcmd*{\switcht@albion}{ \def}{\def}{}{}
\xpatchcmd{\switcht@albion}{\relax}{}{}{}
\edef\switcht@deutsch{%
  \relax\unexpanded\expandafter{\switcht@deutsch}%
}
\xpatchcmd*{\switcht@deutsch}{ \def}{\def}{}{}
\xpatchcmd{\switcht@deutsch}{\relax}{}{}{}
\edef\switcht@francais{%
  \relax\unexpanded\expandafter{\switcht@francais}%
}
\xpatchcmd*{\switcht@francais}{ \def}{\def}{}{}
\xpatchcmd{\switcht@francais}{\relax}{}{}{}
\makeatother

\usepackage{ifluatex}
\ifluatex
  \usepackage{fontspec}
  \usepackage[english]{selnolig}
\fi

\iftrue % use default-font
  \ifluatex
    % use the better (sharper, ...) Latin Modern variant of Computer Modern
    \setmainfont{Latin Modern Roman}
    \setsansfont{Latin Modern Sans}
    \setmonofont{Latin Modern Mono} % "variable=false"
    %\setmonofont{Latin Modern Mono Prop} % "variable=true"
  \else
    % better font, similar to the default springer font
    % cfr-lm is preferred over lmodern. Reasoning at http://tex.stackexchange.com/a/247543/9075
    \usepackage[%
      rm={oldstyle=false,proportional=true},%
      sf={oldstyle=false,proportional=true},%
      tt={oldstyle=false,proportional=true,variable=false},%
      qt=false%
    ]{cfr-lm}
  \fi
\else
  % In case more space is needed, it is accepted to use Times New Roman
  \ifluatex
    \setmainfont{TeX Gyre Termes}
    \setsansfont[Scale=.9]{TeX Gyre Heros}
    % newtxtt looks good with times, but no equivalent for lualatex found,
    % therefore tried to replace with inconsolata.
    % However, inconsolata does not look good in the context of LNCS ...
    %\setmonofont[StylisticSet={1,3},Scale=.9]{inconsolata}
    % ... thus, we use the good old Latin Modern Mono font for source code.
    \setmonofont{Latin Modern Mono} % "variable=false"
    %\setmonofont{Latin Modern Mono Prop} % "variable=true"
  \else
    % overwrite cmodern with the Times variant
    \usepackage{newtxtext}
    \usepackage{newtxmath}
    \usepackage[zerostyle=b,scaled=.9]{newtxtt}
  \fi
\fi

\ifluatex
\else
  % fontenc and inputenc are not required when using lualatex
  \usepackage[T1]{fontenc}
  \usepackage[utf8]{inputenc} %support umlauts in the input
\fi

\usepackage{graphicx}

% backticks (`) are rendered as such in verbatim environment. See https://tex.stackexchange.com/a/341057/9075 for details.
\usepackage{upquote}

% Nicer tables (\toprule, \midrule, \bottomrule - see example)
\usepackage{booktabs}

%extended enumerate, such as \begin{compactenum}
\usepackage{paralist}

%put figures inside a text
%\usepackage{picins}
%use
%\piccaptioninside
%\piccaption{...}
%\parpic[r]{\includegraphics ...}
%Text...

% For easy quotations: \enquote{text}
% This package is very smart when nesting is applied, otherwise textcmds (see below) provides a shorter command
\usepackage{csquotes}

% For even easier quotations: \qq{text}
\usepackage{textcmds}

%enable margin kerning
\RequirePackage[%
  babel,%
  final,%
  expansion=alltext,%
  protrusion=alltext-nott]{microtype}%
% \texttt{test -- test} keeps the "--" as "--" (and does not convert it to an en dash)
\DisableLigatures{encoding = T1, family = tt* }

%tweak \url{...}
\usepackage{url}
%\urlstyle{same}
%improve wrapping of URLs - hint by http://tex.stackexchange.com/a/10419/9075
\makeatletter
\g@addto@macro{\UrlBreaks}{\UrlOrds}
\makeatother
%nicer // - solution by http://tex.stackexchange.com/a/98470/9075
%DO NOT ACTIVATE -> prevents line breaks
%\makeatletter
%\def\Url@twoslashes{\mathchar`\/\@ifnextchar/{\kern-.2em}{}}
%\g@addto@macro\UrlSpecials{\do\/{\Url@twoslashes}}
%\makeatother

% Diagonal lines in a table - http://tex.stackexchange.com/questions/17745/diagonal-lines-in-table-cell
% Slashbox is not available in texlive (due to licensing) and also gives bad results. This, we use diagbox
%\usepackage{diagbox}

% Required for package pdfcomment later
\usepackage{xcolor}

% For listings
\usepackage{listings}
\lstset{%
  basicstyle=\ttfamily,%
  columns=fixed,%
  basewidth=.5em,%
  xleftmargin=0.5cm,%
  captionpos=b}%
\renewcommand{\lstlistingname}{List.}
% Fix counter as described at https://tex.stackexchange.com/a/28334/9075
\usepackage{chngcntr}
\AtBeginDocument{\counterwithout{lstlisting}{section}}

% Enable nice comments
\usepackage{pdfcomment}
%
\newcommand{\commentontext}[2]{\colorbox{yellow!60}{#1}\pdfcomment[color={0.234 0.867 0.211},hoffset=-6pt,voffset=10pt,opacity=0.5]{#2}}
\newcommand{\commentatside}[1]{\pdfcomment[color={0.045 0.278 0.643},icon=Note]{#1}}
%
% Compatibality with packages todo, easy-todo, todonotes
\newcommand{\todo}[1]{\commentatside{#1}}
% Compatiblity with package fixmetodonotes
\newcommand{\TODO}[1]{\commentatside{#1}}

% Bibliopgraphy enhancements
%  - enable \cite[prenote][]{ref}
%  - enable \cite{ref1,ref2}
% Alternative: \usepackage{cite}, which enables \cite{ref1, ref2} only (otherwise: Error message: "White space in argument")

\ifluatex
  % does not work when using luatex
  % see: https://tex.stackexchange.com/q/419288/9075
\else
  % Prepare more space-saving rendering of the bibliography
  % Source: https://tex.stackexchange.com/a/280936/9075
  \SetExpansion
  [ context = sloppy,
    stretch = 30,
    shrink = 60,
    step = 5 ]
  { encoding = {OT1,T1,TS1} }
  { }
\fi

% Put footnotes below floats
% Source: https://tex.stackexchange.com/a/32993/9075
\usepackage{stfloats}
\fnbelowfloat

% Enable that parameters of \cref{}, \ref{}, \cite{}, ... are linked so that a reader can click on the number an jump to the target in the document
\usepackage{hyperref}
% Enable hyperref without colors and without bookmarks
\hypersetup{hidelinks,
  colorlinks=true,
  allcolors=black,
  pdfstartview=Fit,
  breaklinks=true}
%
% Enable correct jumping to figures when referencing
\usepackage[all]{hypcap}

\usepackage[group-four-digits,per-mode=fraction]{siunitx}

%enable \cref{...} and \Cref{...} instead of \ref: Type of reference included in the link
\usepackage[capitalise,nameinlink]{cleveref}
%Nice formats for \cref
\usepackage{iflang}
\IfLanguageName{ngerman}{
  \crefname{table}{Tab.}{Tab.}
  \Crefname{table}{Tabelle}{Tabellen}
  \crefname{figure}{\figurename}{\figurename}
  \Crefname{figure}{Abbildungen}{Abbildungen}
  \crefname{equation}{Gleichung}{Gleichungen}
  \Crefname{equation}{Gleichung}{Gleichungen}
  \crefname{listing}{\lstlistingname}{\lstlistingname}
  \Crefname{listing}{Listing}{Listings}
  \crefname{section}{Abschnitt}{Abschnitte}
  \Crefname{section}{Abschnitt}{Abschnitte}
  \crefname{paragraph}{Abschnitt}{Abschnitte}
  \Crefname{paragraph}{Abschnitt}{Abschnitte}
  \crefname{subparagraph}{Abschnitt}{Abschnitte}
  \Crefname{subparagraph}{Abschnitt}{Abschnitte}
}{
  \crefname{section}{Sect.}{Sect.}
  \Crefname{section}{Section}{Sections}
  \crefname{listing}{\lstlistingname}{\lstlistingname}
  \Crefname{listing}{Listing}{Listings}
}


%Intermediate solution for hyperlinked refs. See https://tex.stackexchange.com/q/132420/9075 for more information.
\newcommand{\Vlabel}[1]{\label[line]{#1}\hypertarget{#1}{}}
\newcommand{\lref}[1]{\hyperlink{#1}{\FancyVerbLineautorefname~\ref*{#1}}}

\usepackage{xspace}
%\newcommand{\eg}{e.\,g.\xspace}
%\newcommand{\ie}{i.\,e.\xspace}
\newcommand{\eg}{e.\,g.,\ }
\newcommand{\ie}{i.\,e.,\ }

%introduce \powerset - hint by http://matheplanet.com/matheplanet/nuke/html/viewtopic.php?topic=136492&post_id=997377
\DeclareFontFamily{U}{MnSymbolC}{}
\DeclareSymbolFont{MnSyC}{U}{MnSymbolC}{m}{n}
\DeclareFontShape{U}{MnSymbolC}{m}{n}{
  <-6>    MnSymbolC5
  <6-7>   MnSymbolC6
  <7-8>   MnSymbolC7
  <8-9>   MnSymbolC8
  <9-10>  MnSymbolC9
  <10-12> MnSymbolC10
  <12->   MnSymbolC12%
}{}
\DeclareMathSymbol{\powerset}{\mathord}{MnSyC}{180}

\ifluatex
\else
  % Enable copy and paste - also of numbers
  % This has to be done instead of \usepackage{cmap}, because it does not work together with cfr-lm.
  % See: https://tex.stackexchange.com/a/430599/9075
  \input glyphtounicode
  \pdfgentounicode=1
\fi

% correct bad hyphenation here
\hyphenation{op-tical net-works semi-conduc-tor}

%% END COPYING HERE


% Add copyright
% Do that for the final version or if you send it to colleagues
\iffalse
  %state: intended|submitted|llncs
  %you can add "crop" if the paper should be cropped to the format Springer is publishing
  \usepackage[intended]{llncsconf}

  \conference{name of the conference}

  %in case of "llncs" (final version!)
  %example: llncs{Anonymous et al. (eds). \emph{Proceedings of the International Conference on \LaTeX-Hacks}, LNCS~42. Some Publisher, 2016.}{0042}
  \llncs{book editors and title}{0042} %% 0042 is the start page
\fi

% For demonstration purposes only
\usepackage[math]{blindtext}
\usepackage{mwe}
\usepackage[backend=biber, style=numeric]{biblatex}
\addbibresource{java.bib}

\begin{document}

\title{Procesos Importantes de la Máquina Virtual de Java}

\author{Laura Pérez Medeiro}
\institute{Universidad de Alcala}

\maketitle

\abstract{La Máquina Virtual de Java (JVM) es uno de los sistemas de ejecución
de código más empleados hoy en día. En el mundo de la programación, el esquema
de ejecución por máquina virtual es uno de los más famosos, ya que este método tiene
muchas de las ventajas de la ejecucicón en máquinas virtuales, como no estar
ligado a ningún sistema operativo en particular, a la vez que disfruta de
características como alta seguridad o buen rendimiento a la hora de ejecutar
código. Este artículo explorará los procesos más importantes de la JVM y cómo
esta consigue transformar código Java (y no solo Java) en un programa
ejecutable.}

\section*{La Máquina Virtual de Java: Una perspectiva de alto nivel}
Al analizar el gran abanico de herramientas que es el lenguaje de programación
Java, el lector comprenderá de inmediato que el punto central es la Máquina
Virtual de Java. Las características más importantes que se pueden encontrar en
Java se deben a la existencia de la JVM. Desde la independencia
\textit{hardware} y de sistema operativo, hasta el pequeño tamaño de su código
compilado, la JVM está involucrada de alguna manera u otra.

Si se analizara la JVM por dentro esta podría ser descrita como una máquina de
computación abstracta (o \textit{Abstract Computing Machine}). Como ya se ha
mencionado, hoy en día, implementar un lenguaje de programación en base a una
máquina virtual no es algo extraordinario, y algunos de los lenguajes con mejor
rendimiento están implementados de esta maner.

La primera vez que se implementó la Máquina Virtual de Java con éxito, apenas
emulaba el set de instrucciones de la JVM en una antigua PDA. Desde entonces,
las implementaciones más modernas han sido capaces de ejecutar funcionar en
prácticamente cualquier tipo de plataforma imaginable. De hecho, esa es una de
las características más atractivas de JVM, la máquina virtual no está diseñada
para ninguna arquitectura en concreto y puede ejecutarse en la enorme mayoría de
sistemas operativos modernos; de la misma manera podría ser implementada en un
chip físico, si esto fuera útil para el implementador.

Aunque sea el punto central del lenguaje de programación Java, la JVM conoce
prácticamente nada del lenguaje mismo, ya que esta funciona completamente en
base al formato de archivo llamado \textit{class}. Este formato contiene
instrucciones de ejecución para la JVM, una tabla de símbolos y otra información
útil.

La JVM crea proactivamente límites sobre la estructura y la sintaxis del código
de los archivos \textit{class}. Gracias a esto, cualquier lenguaje de
programación que pueda ser expresado en un archivo \textit{class} podrá ser
ejecutado en la JVM. La filosofía general de la JVM es ser un intermediario
generalista e independiente a la arquitectura de la máquina para cualquier
lenguaje de programación que quiera hacer uso de ella.

\section*{De Java a 1s y 0s: \textit{bytecode}}
Por dentro, la JVM es una máquina de pila (aunque también emplee registros).
Dentro de la ejecución del código, cada marco para un método emplea una pila de
operandos y un array de variables locales. El array de operandos es usado para
computaciones y para recibir valores de retorno, mientras que las variables
locales tienen el mismo fin que los registros. El tamaño máximo para la pila de
operandos es parte de los atributos de cada método y generalmente se suele
encontrar entre 0 y 4294967295, donde cada valor tiene 32 bits.

Como ya se mencionó anteriormente, los \textit{bytecodes} son lenguaje con el
que la JVM funciona. Cuando la JVM carga un archivo \textit{class}, se encarga
de coger los \textit{streams} de bytecodes de cada método individual dentro de
la clase a la que pertenecen, y los guarda en el área de método (o
\textit{method area}). Estos streams pueden ser ejecutados con la misma
ejecución misma del programa, y se ejecutarán mediante interpretación,
compilación \textit{just-in-time}, o cualquier otro método escogido por el
implementador de la JVM en concreto.

Con objetivo de aclarar el concepto, el \textit{stream} de \textit{bytecode} de
un método en concreto, es la secuencia de instrucciones de la Máquina Virtual de
Java que esta debe seguir precisa y exactamente. Cada instrucción contiene un
\textit{opcode} de 8 bits, posiblemente seguido de al menos un operando (e
información extra si fuera necesaria por la instrucción en concreto que la JVM
estuviera ejecutando en ese momento).

El set de instrucciones del \textit{bytecode} fue diseñado para ser
extremadamente compacto. Por ese mismo motivo, la enorme mayoría de
instrucciones se alinean con los límites de un byte. Es decir, están diseñadas
teniendo en cuenta la eliminación de codificacion extra o innecesaria como pilar
principal. Gracias a esto mismo se consigue el el número de \textit{opcodes} sea
lo suficientemente pequeño como para que se pueda notar una reducción
significativa en el tamaño de los archivos que la JVM emplea.

Aunque la JVM sea técnicamente un autómata de pila y de registros a la vez, esta
no dispone de registros que le permitan guardar valores arbitrarios, así que
todo deberá ser colocado en la pila antes de que se pueda usar en un cálculo. La
JVM fue diseñada para ser híbrida por naturaleza, pero se centró en el aspecto
de la pila para facilitar la implementación de la misma en arquitecturas que no
disponían de un gran abanico de registros para sus operaciones. Por tanto, es
importante mencionar que aunque la JVM sea un autómata híbrido (de pila y
registros), los\textit{bytecodes} operan principalmente en la pila.

\subsection*{Tipos}
Just like the Java programming language, the JVM operates on two different kinds
of types: \textit{primitive types} and \textit{reference types}\footnote{Se
puede decir que los tipos de referencia son prácticamente el sistema de
objectos/punteros a objetos de otros lenguajes de programación}.

Al igual que el lenguaje de programación Java, la JVM opera con tipos primitivos
y tipos de referencia. Esta espera que prácticamente todas las comprobaciones
sean realizadas antes del tiempo de ejecución (generalmente por el compilador) y
que la responsabilidad no recae en la JVM misma. Por ese mismo motivo, los
valores primitivos no tienen que estar marcados con un identificador que los
diferencie por tipo del resto de los valores en tiempo de ejecución. Lo que la
JVM hace es crear diferentes series de variaciones en base a la misma operación
que se adaptan a los tipos con los que tengan que cuadrar. Explicado de una
manera un poco más simple: para la operación de tipo X que puede ejecutar la
JVM, esta creará variaciones para cada tipo de primitiva dentro de la operación
X y, en vez de tener que identificar los tipos a la hora de aplicar la
operación, simplemente usa la variación correspondiente.

\subsubsection*{Tipos numéricos}
The numeric types consist of the \textit{integral types} and the
\textit{floating point types}.

Los tipos numéricos consisten de \textit{tipos integrales} y de los
\textit{tipos de punto flotante}.

Los tipos integrales son los siguientes: \textit{\textbf{byte}},
\textit{\textbf{short}}, \textit{\textbf{int}} y
\textit{\textbf{long}}. Estos son valores numéricos de 8, 16, 32 y 64 bits
respectivamente en complemento a dos, y cuyo valor por defecto es
0. Sus valores van entre \(-2^{num bits-1}\) y \(2^{num bits-1}\)
inclusivamente. \parencite{jvmspec}

Dentro de los tipos integrales también existe el tipo \textit{char}, cuyos
valores son numéricos de 16 bits en binario puro representando los valores
\textit{Unicode} y codificados en \textit{UTF-16}. El rango de valores aceptados
está entre el 0 y el 65535 incluído.

Los tipos de punto flotante son: \textbf{\textit{float}} y
\textbf{\textit{double}} y son elementos del \textit{value set} de cada uno
correspondientemente, y cuyo valor por defecto es cero.

\section*{La creación de un ejecutable}
El proceso de ejecución del compilador de Java (\textit{javac} generalmente), un
archivo \textit{class} es creado. Como bien podrá comprender el lector, ese
archivo aún está lejos de poder ser ejecutado por el ordenador, y tendrá que
verse sometido a una serie de procesos para convertir lo que antes era código
Java en instrucciones para el procesador. El archivo aún es totalmente
dependiente del JVM, y gracias a este el archivo dispone de una plataforma donde
ejecutar sus acciones. La máquina virtual de Java actúa de intermediario entre
el programa y el ordenador y controla los recursos de este mismo.

Esto no ocurre automáticamente, si se analizan internamente los procesos que la
JVM lleva a cabo, se pueden encontrar rápidamente tres que resaltan por enima de
los demás. Estos se llaman \textit{carga, enlazado e inicialización}.

\subsection*{Carga}
El proceso de carga o \textit{loading} se define de una manera muy simple. Según
la especificación online de la máquina virtual de Java \parencite{jvmspec}. Este
es el \textit{proceso de encontrar la representación binaria de una clase o
interfaz con un nombre en concreto y crear una clase o interfaz en base a esa
representación binaria}.

Hay dos tipos principales de cargadores de clases de la JVM: (1) el
cargador de clases \textit{bootstrap} y (2) el cargador de clases definido por
el usuario. El cargador \textit{bootstrap} es el cargador por defecto,
estrictamente definido por la especificación estándar de la JVM, y carga los
archivos según la misma. Por otra parte, el cargador definido por el usuario
está diseñado para que su implementación pueda ser alterada y carga las clases
a través de la instancia de \textit{java.lang.Class}.

Normalmente, un cargador de clases guarda la representación binaria de aquello
que va a cargar antes de que esta haga falta. De esta manera, se puede asegurar
tenerla preparada para el momento en el que la tenga que emplear, y al tenerla
cargada pueda mejorar el rendimiento del proceso de cargado. Cabe la pena
mencionar que si cualquier error se encuentra en las fases iniciales del proceso
de cargado, la JVM espera hasta que la clase o interfaz que contiene ese error
sea invocado para reportar la existencia del mismo. Si no se invoca o referencia
la clase que contiene el error, este podrá persistir y JVM no notificará de su
existencia, aunque sea consciente de ella.

Por tanto, se podría decir que el proceso de carga se resume en los siguientes
pasos principales: (1) creación de un \textit{stream} binario de datos desde el archivo
\textit{class} y (2) \textit{parseo}.

\subsection*{Enlazado}
Según la especificación online del JVM, el proceso de enlazado es \textit{el
proceso en el que se combina una clase o interfaz con el estado de ejecución de
la JVM para que pueda ser ejecutado}.

Este proceso empieza con una fase llamada \textit{verificación} de la clase.
Aquí es donde la JVM se asegura de que el código siga los criterior sintácticos
del lenguaje y que añadirlo al entorno de ejecución no vaya a crear ninguna
disrupción en el estado de la JVM. Aunque este proceso está estrictamente
definido y estandarizado por la JVM, la especificación sigue siendo lo
suficientemente flexible como para que esta sea alterada por los diferentes
implementadores y estos puedan decidir cómo deberían ejecutarse los diferentes
procesos de enlazado.

Entre otros procesos que ocurren en la fase de enlazados se pueden encontrar
todas las pequeñas excepciones que pueden ocurrir debido a la extensiva lista de
casos indeseados que la JVM especifica. La máquina virtual de Java completa
estas comprobaciones desde el principio y se asegura que pequeños errores no
detectados desde un buen comienzo no hagan que el programa vaya a
\textit{crashear}. También se hacen diferentes comprobaciones para asegurarse
que la estrucutra de los datos binarios se alinean con el formato esperado.
Aunque haya múltiples verificaciones a lo largo de los diferentes pasos,
generalmente se considera que los procesos de verificación empiezan realmente en
la fase de enlazado.

Una vez que los procesos de verificación de la fase de enlazado se han compleato
con éxito, la JVM reserva memoria para las variables de clase y las inicializa
correspondiendo con sus respectivos tipos. Cabe mencionar que esta no es la
inicialización \textit{real} (con los valores definidos por el usuario) y que
esta no ocurre hasta la siguiente fase de inicialización. Este proceso se
denomina como \textit{preparación}.

Al final del todo, en la fase \textbf{opcional} de resolución, la JVM localiza
las referencias de cualquier clase, campo o método en la tabla de símbolos y
determina sus valores reales. Al igual que muchas otras características, este
proceso también puede ser customizado por los implementadores de la máquina
virtual para que siga los pasos específicos que estos deseen.

\subsection*{Inicialización}
Al igual que en los apartados anteriores, se empleará la especificación oficial
de la JVM. En esta se describe el proceso de inicialización como \textit{la
ejecución de el método de inicialización de una clase o interfaz}.

Este proceso es extremadamente simple, dado que lo que realmente se consigue es
que la clase o interfaz esté preparada para su uso real. El proceso comienza con
la inicialización de las variables de clase con los valores especificados. Esto
significa que las variables de clase son inicializadas por medio de alguna
rutina de inicialización escrita por el programador. También hay que tener en
cuenta la inicialización de las superclases si estas no lo hubieran sido con
anterioridad. 

\section*{Características importantes}
Como ya mencionó al principio de este artículo, el uso de ejecución en base a
máquina virtual para lenguajes de programación tiene tanto sus ventajas como
desventajas. Esta sección se comentarán y analizarán las características más
importantes de la máquina virtual de Java.

\subsection*{Independencia de plataforma}
Una de las características más importantes de la máquina virtual de Java es su
extrema independencia de la plataforma en la que se ejecuta. La JVM tiene la
habilidad de ser instalada en prácticamente cualquier sistema operativo y
arquitectura y funcionar de la misma manera. De hecho, hoy en día la JVM puede
ejecutarse desde sistemas empotrados hasta dispositivos móviles. Esta
transformará los programas a \textit{bytecode} igualmente.

Hoy en día Java corre en miles de millones de dispositivos, y no sería
descabellado pensar que su independencia de la plataforma es uno de los
factores que han podido influenciar su desarrollo de esa manera. 

\subsection*{Seguridad}
Uno de los pilares de la filosofía de JVM mientras se diseñaba inicialmente era
la seguridad. La JVM hace escaneos exhaustivos del bytecode que va a ejecutar
para poder asegurar que este no contenga ninguna instrucción posiblemente
maliciosa o no segura, como acceso a lugares prohibidos dentro de la memoria,
operaciones privilegiadas, etc.\ También hay una serie de reglas que ningún
programa puede romper mientras se ejecuta, lo que consigue un nivel de seguridad
superior: si por cualquier casual superara los checkeos pre-ejecución, es
prácticamente seguro que se capture el error en tiempo de ejecución. Esto es
especialmente efectivo para la prevención de errores relacionados con la
corrupión de memoria (los cuales son muy frecuentes en lenguajes como C, que no
realiza ningún checkeo de este estilo).

También se prohibe la realización de aritmética de punteros y \textit{casteo} de
variables sin precomprobación de que este funcione. Asimismo, no se permite la
reserva y liberación de memoria de manera manual; el usuario del lenguaje deberá
basarse en el recolector de basura por defecto para esas labores.

\subsection*{Velocidad}
El hecho de emplear una máquina virtual es a la vez ventaja y desventaja para
Java. En el aspecto velocidad, cualquier máquina virtual será más lenta que un
lenguaje compilado directamente, por definición. Aunque esto no significa que la JVM
sea radicalmente más diferente que los lenguajes basados en C.

En los inicicios de Java sus tiempos de ejecución y su rendimiento eran factores
de alta preocupación entre sus usuarios. Desde entonces, las diferentes
versiones y optimizaciones han conseguido que, incluso siendo un lenguaje
basado en máquina virtual, los lengujes como C le saquen muy poca ventaja
respecto a tiempos de ejecución. De hecho, Java sigue siendo uno de los
lenguajes con más rendimiento de hoy en día.

El \textit{bytecode} de Java puede ser interpretado en tiempo de ejecución por
la máquina virtual o compilado en código nativo que corre directamente en el
hardware de la máquina. La interpretación siempre es más lenta que la
compilación, por lo que la mayoría de implementaciones modernas de la JVM que se
precien emplearán compilación por defecto. Aunque este método tenga un
\textit{overhead} inicial por la necesidad de compilar el propio código, una vez
compilado Java tien prácticamente la misma (por no decir más) velocidad que 
lenguajes modernos decaracterísticas similares.

\subsection*{Imposibilidad de cambiar características definidas por el sistema}
La máquina virtual es muy flexible a la hora de adaptarse al sistema y a la
arquitectura en la que esta se ejecute. Aun así, hay ciertas características por
defecto de estos mismos que simplemente no se pueden cambiar. Un claro ejemplo
es el de las interfaces gráficas de usuario: por mucho que la JVM defina un
\textit{look and feel} predeterminados, si el sistema operativo fuerza su propia
interfaz a todos los programas, JVM no tendrá más opción que obedecer y ejecutar
con esos parámetros por defecto.

Hay que recordar que JVM sigue siendo un mero programa que corre en un sistema
operativo, independientemente de que este sirva para que otros programas se
ejecuten en base a él.

\newpage
\printbibliography

\end{document}
