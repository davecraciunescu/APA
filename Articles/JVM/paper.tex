% This template has been tested with LLNCS DOCUMENT CLASS -- version 2.20 (10-Mar-2018)

% !TeX spellcheck = en-US
% !TeX encoding = utf8
% !TeX program = pdflatex
% !BIB program = bibtex
% -*- coding:utf-8 mod:LaTeX -*-

% "a4paper" enables:
%  - easy print out on DIN A4 paper size
%
% One can configure a4 vs. letter in the LaTeX installation. So it is configuration dependend, what the paper size will be.
% This option  present, because the current word template offered by Springer is DIN A4.
% We accept that DIN A4 cause WTFs at persons not used to A4 in USA.

%/ "runningheads" enables:
%  - page number on page 2 onwards
%  - title/authors on even/odd pages
% This is good for other readers to enable proper archiving among other papers and pointing to
% content. Even if the title page states the title, when printed and stored in a folder, when
% blindly opening the folder, one could hit not the title page, but an arbitrary page. Therefore,
% it is good to have title printed on the pages, too.
%
% It is enabled by default as the springer template as of 2018/03/10 uses this as default

% German documents: pass ngerman as class option
% \documentclass[ngerman,runningheads,a4paper]{llncs}[2018/03/10]
% English documents: pass english as class option
\documentclass[english,runningheads,a4paper]{llncs}[2018/03/10]

%% If you need packages for other papers,
%% START COPYING HERE

% Set English as language and allow to write hyphenated"=words
%
% In case you write German, switch the parameters, so that the command becomes
%\usepackage[english,main=ngerman]{babel}
%
% Even though `american`, `english` and `USenglish` are synonyms for babel package (according to https://tex.stackexchange.com/questions/12775/babel-english-american-usenglish), the llncs document class is prepared to avoid the overriding of certain names (such as "Abstract." -> "Abstract" or "Fig." -> "Figure") when using `english`, but not when using the other 2.
% english has to go last to set it as default language
\usepackage[ngerman,main=english]{babel}
%
% Hint by http://tex.stackexchange.com/a/321066/9075 -> enable "= as dashes
\addto\extrasenglish{\languageshorthands{ngerman}\useshorthands{"}}
%
% Fix by https://tex.stackexchange.com/a/441701/9075
\usepackage{regexpatch}
\makeatletter
\edef\switcht@albion{%
  \relax\unexpanded\expandafter{\switcht@albion}%
}
\xpatchcmd*{\switcht@albion}{ \def}{\def}{}{}
\xpatchcmd{\switcht@albion}{\relax}{}{}{}
\edef\switcht@deutsch{%
  \relax\unexpanded\expandafter{\switcht@deutsch}%
}
\xpatchcmd*{\switcht@deutsch}{ \def}{\def}{}{}
\xpatchcmd{\switcht@deutsch}{\relax}{}{}{}
\edef\switcht@francais{%
  \relax\unexpanded\expandafter{\switcht@francais}%
}
\xpatchcmd*{\switcht@francais}{ \def}{\def}{}{}
\xpatchcmd{\switcht@francais}{\relax}{}{}{}
\makeatother

\usepackage{ifluatex}
\ifluatex
  \usepackage{fontspec}
  \usepackage[english]{selnolig}
\fi

\iftrue % use default-font
  \ifluatex
    % use the better (sharper, ...) Latin Modern variant of Computer Modern
    \setmainfont{Latin Modern Roman}
    \setsansfont{Latin Modern Sans}
    \setmonofont{Latin Modern Mono} % "variable=false"
    %\setmonofont{Latin Modern Mono Prop} % "variable=true"
  \else
    % better font, similar to the default springer font
    % cfr-lm is preferred over lmodern. Reasoning at http://tex.stackexchange.com/a/247543/9075
    \usepackage[%
      rm={oldstyle=false,proportional=true},%
      sf={oldstyle=false,proportional=true},%
      tt={oldstyle=false,proportional=true,variable=false},%
      qt=false%
    ]{cfr-lm}
  \fi
\else
  % In case more space is needed, it is accepted to use Times New Roman
  \ifluatex
    \setmainfont{TeX Gyre Termes}
    \setsansfont[Scale=.9]{TeX Gyre Heros}
    % newtxtt looks good with times, but no equivalent for lualatex found,
    % therefore tried to replace with inconsolata.
    % However, inconsolata does not look good in the context of LNCS ...
    %\setmonofont[StylisticSet={1,3},Scale=.9]{inconsolata}
    % ... thus, we use the good old Latin Modern Mono font for source code.
    \setmonofont{Latin Modern Mono} % "variable=false"
    %\setmonofont{Latin Modern Mono Prop} % "variable=true"
  \else
    % overwrite cmodern with the Times variant
    \usepackage{newtxtext}
    \usepackage{newtxmath}
    \usepackage[zerostyle=b,scaled=.9]{newtxtt}
  \fi
\fi

\ifluatex
\else
  % fontenc and inputenc are not required when using lualatex
  \usepackage[T1]{fontenc}
  \usepackage[utf8]{inputenc} %support umlauts in the input
\fi

\usepackage{graphicx}

% backticks (`) are rendered as such in verbatim environment. See https://tex.stackexchange.com/a/341057/9075 for details.
\usepackage{upquote}

% Nicer tables (\toprule, \midrule, \bottomrule - see example)
\usepackage{booktabs}

%extended enumerate, such as \begin{compactenum}
\usepackage{paralist}

%put figures inside a text
%\usepackage{picins}
%use
%\piccaptioninside
%\piccaption{...}
%\parpic[r]{\includegraphics ...}
%Text...

% For easy quotations: \enquote{text}
% This package is very smart when nesting is applied, otherwise textcmds (see below) provides a shorter command
\usepackage{csquotes}

% For even easier quotations: \qq{text}
\usepackage{textcmds}

%enable margin kerning
\RequirePackage[%
  babel,%
  final,%
  expansion=alltext,%
  protrusion=alltext-nott]{microtype}%
% \texttt{test -- test} keeps the "--" as "--" (and does not convert it to an en dash)
\DisableLigatures{encoding = T1, family = tt* }

%tweak \url{...}
\usepackage{url}
%\urlstyle{same}
%improve wrapping of URLs - hint by http://tex.stackexchange.com/a/10419/9075
\makeatletter
\g@addto@macro{\UrlBreaks}{\UrlOrds}
\makeatother
%nicer // - solution by http://tex.stackexchange.com/a/98470/9075
%DO NOT ACTIVATE -> prevents line breaks
%\makeatletter
%\def\Url@twoslashes{\mathchar`\/\@ifnextchar/{\kern-.2em}{}}
%\g@addto@macro\UrlSpecials{\do\/{\Url@twoslashes}}
%\makeatother

% Diagonal lines in a table - http://tex.stackexchange.com/questions/17745/diagonal-lines-in-table-cell
% Slashbox is not available in texlive (due to licensing) and also gives bad results. This, we use diagbox
%\usepackage{diagbox}

% Required for package pdfcomment later
\usepackage{xcolor}

% For listings
\usepackage{listings}
\lstset{%
  basicstyle=\ttfamily,%
  columns=fixed,%
  basewidth=.5em,%
  xleftmargin=0.5cm,%
  captionpos=b}%
\renewcommand{\lstlistingname}{List.}
% Fix counter as described at https://tex.stackexchange.com/a/28334/9075
\usepackage{chngcntr}
\AtBeginDocument{\counterwithout{lstlisting}{section}}

% Enable nice comments
\usepackage{pdfcomment}
%
\newcommand{\commentontext}[2]{\colorbox{yellow!60}{#1}\pdfcomment[color={0.234 0.867 0.211},hoffset=-6pt,voffset=10pt,opacity=0.5]{#2}}
\newcommand{\commentatside}[1]{\pdfcomment[color={0.045 0.278 0.643},icon=Note]{#1}}
%
% Compatibality with packages todo, easy-todo, todonotes
\newcommand{\todo}[1]{\commentatside{#1}}
% Compatiblity with package fixmetodonotes
\newcommand{\TODO}[1]{\commentatside{#1}}

% Bibliopgraphy enhancements
%  - enable \cite[prenote][]{ref}
%  - enable \cite{ref1,ref2}
% Alternative: \usepackage{cite}, which enables \cite{ref1, ref2} only (otherwise: Error message: "White space in argument")

\ifluatex
  % does not work when using luatex
  % see: https://tex.stackexchange.com/q/419288/9075
\else
  % Prepare more space-saving rendering of the bibliography
  % Source: https://tex.stackexchange.com/a/280936/9075
  \SetExpansion
  [ context = sloppy,
    stretch = 30,
    shrink = 60,
    step = 5 ]
  { encoding = {OT1,T1,TS1} }
  { }
\fi

% Put footnotes below floats
% Source: https://tex.stackexchange.com/a/32993/9075
\usepackage{stfloats}
\fnbelowfloat

% Enable that parameters of \cref{}, \ref{}, \cite{}, ... are linked so that a reader can click on the number an jump to the target in the document
\usepackage{hyperref}
% Enable hyperref without colors and without bookmarks
\hypersetup{hidelinks,
  colorlinks=true,
  allcolors=black,
  pdfstartview=Fit,
  breaklinks=true}
%
% Enable correct jumping to figures when referencing
\usepackage[all]{hypcap}

\usepackage[group-four-digits,per-mode=fraction]{siunitx}

%enable \cref{...} and \Cref{...} instead of \ref: Type of reference included in the link
\usepackage[capitalise,nameinlink]{cleveref}
%Nice formats for \cref
\usepackage{iflang}
\IfLanguageName{ngerman}{
  \crefname{table}{Tab.}{Tab.}
  \Crefname{table}{Tabelle}{Tabellen}
  \crefname{figure}{\figurename}{\figurename}
  \Crefname{figure}{Abbildungen}{Abbildungen}
  \crefname{equation}{Gleichung}{Gleichungen}
  \Crefname{equation}{Gleichung}{Gleichungen}
  \crefname{listing}{\lstlistingname}{\lstlistingname}
  \Crefname{listing}{Listing}{Listings}
  \crefname{section}{Abschnitt}{Abschnitte}
  \Crefname{section}{Abschnitt}{Abschnitte}
  \crefname{paragraph}{Abschnitt}{Abschnitte}
  \Crefname{paragraph}{Abschnitt}{Abschnitte}
  \crefname{subparagraph}{Abschnitt}{Abschnitte}
  \Crefname{subparagraph}{Abschnitt}{Abschnitte}
}{
  \crefname{section}{Sect.}{Sect.}
  \Crefname{section}{Section}{Sections}
  \crefname{listing}{\lstlistingname}{\lstlistingname}
  \Crefname{listing}{Listing}{Listings}
}


%Intermediate solution for hyperlinked refs. See https://tex.stackexchange.com/q/132420/9075 for more information.
\newcommand{\Vlabel}[1]{\label[line]{#1}\hypertarget{#1}{}}
\newcommand{\lref}[1]{\hyperlink{#1}{\FancyVerbLineautorefname~\ref*{#1}}}

\usepackage{xspace}
%\newcommand{\eg}{e.\,g.\xspace}
%\newcommand{\ie}{i.\,e.\xspace}
\newcommand{\eg}{e.\,g.,\ }
\newcommand{\ie}{i.\,e.,\ }

%introduce \powerset - hint by http://matheplanet.com/matheplanet/nuke/html/viewtopic.php?topic=136492&post_id=997377
\DeclareFontFamily{U}{MnSymbolC}{}
\DeclareSymbolFont{MnSyC}{U}{MnSymbolC}{m}{n}
\DeclareFontShape{U}{MnSymbolC}{m}{n}{
  <-6>    MnSymbolC5
  <6-7>   MnSymbolC6
  <7-8>   MnSymbolC7
  <8-9>   MnSymbolC8
  <9-10>  MnSymbolC9
  <10-12> MnSymbolC10
  <12->   MnSymbolC12%
}{}
\DeclareMathSymbol{\powerset}{\mathord}{MnSyC}{180}

\ifluatex
\else
  % Enable copy and paste - also of numbers
  % This has to be done instead of \usepackage{cmap}, because it does not work together with cfr-lm.
  % See: https://tex.stackexchange.com/a/430599/9075
  \input glyphtounicode
  \pdfgentounicode=1
\fi

% correct bad hyphenation here
\hyphenation{op-tical net-works semi-conduc-tor}

%% END COPYING HERE


% Add copyright
% Do that for the final version or if you send it to colleagues
\iffalse
  %state: intended|submitted|llncs
  %you can add "crop" if the paper should be cropped to the format Springer is publishing
  \usepackage[intended]{llncsconf}

  \conference{name of the conference}

  %in case of "llncs" (final version!)
  %example: llncs{Anonymous et al. (eds). \emph{Proceedings of the International Conference on \LaTeX-Hacks}, LNCS~42. Some Publisher, 2016.}{0042}
  \llncs{book editors and title}{0042} %% 0042 is the start page
\fi

% For demonstration purposes only
\usepackage[math]{blindtext}
\usepackage{mwe}
\usepackage[backend=biber, style=numeric]{biblatex}
\addbibresource{java.bib}

\begin{document}

\title{An analysis of the Java Virtual Machine}

\author{Pablo Acereda García \and David E. Craciunescu}
\institute{Universidad de Alcala}

\maketitle

\abstract{In the world of programming languages, one of the most famous is the
\textit{virtual-machine based} paradigm. This way of designing code compilation
and execution has all the benefits of a fast compilation time, while keeping the
most prominent features of \textit{virtual-machines}, like the ability to run on
any platform and the incomparable level of customization such a framework
enables. In this paper, the most prominent features of the Java Virtual Machine
(JVM) will be explored. The reader is advised, this paper does not treat the
inner architectural mechanisms of the JVM itself, rather it explores the
different processes it carries out to create an executable program, i.e.\ the processing
and creation of the \textit{class} file, and it analyzes advantages and
disadvantages of the framework itself.}

\section*{Introduction}
When analyzing the Java framework, one wouldn't need much time to realize that
the central point of it is the Java Virtual Machine (JVM). Some of Java's most
prominent features are due to this very piece of technology. From hardware and
operating system independence, to the small size of its compiled code, the JVM
is involved in one way or another.

The JVM can be classified as an abstract computing machine. Just as any other
computing machine, it has a specific instruction set and it manipulates
different memory areas during runtime execution. Nowadays, it isn't uncommon to
implement a programming language using a virtual machine. In fact, some of the
best performing programming languages out there are implemented using virtual
machines as well. The most obvious example is the P-Code machine of UCSD Pascal.

The first time the Java Virtual Machine was implemented successfully, it barely
emulated the JVM instruction set in software hosted by a PDA. Since then, the
most modern implementations have been able to run on desktop, server, and even
mobile devices. It is important to note that the Java Virtual Machine does not
assume any specific implementation technology or host hardware. It is
non-inherently interpreted, but can just as well be implemented by compiling the
instruction set to a silicon CPU.

Although the cornerstone of the Java framework, the JVM actually knows nothing
of the Java programming language itself, it only knows about the \textit{class}
file format, which only contains JVM instructions\footnote{Also known as
bytecodes.}, a symbol table and some other useful information.

The JVM enforces strong syntactic and structural constraints on the code of the
\textit{class} file. Thanks to this, any programming language that can be
expressed as a valid \textit{class} file is able to be hosted on the Java
Virtual Machine. The general philosophy of the JVM is to be an available and
machine-independent delivery vehicle for the programming languages that want to
make use of it.

\section*{Instruction set and the class file}
The JVM is both a stack machine and a register machine. Each frame for a method
call has an ``operand stack'' and an array of ``local variables''. The operand
stack is used for operand to computations and for receiving the return value of
a called method, while local variables serve the same purpose as registers and
are also used to pass method arguments. The maximum size of the operand stack
and local variable array, computed by the compiler, is part of the attributes of
each method. Each can be independently size from \textit{0} to \(2^{32} - 1\),
where each value is 32 bits.

\textit{long} and \textit{double} types, which are 64 bits, take up to two
consecutive local variables (which need not be 64-bit aligned in the local
variables array) or one value in the operand stack (but are counted as two units
in the depth of the stack).

\subsection*{The Java bytecode format}
As previously mentioned, \textit{bytecodes} are the machine language of the JVM.
When a JVM loads a \textit{class} file, it gets one stream of bytecodes from
each individual method in the class and stores them in the method area. These
streams are then executed when invoked while running the program. They can be
executed by interpretation, just-in-time compilation, or any other way that was
chosen by the designer of the JVM in which they are being executed in
particular.

A method's \textit{bytecode} stream is the sequence of instructions the Java
Virtual Machine must follow exactly. Each instruction is made up of a one-byte
\textit{opcode} followed by zero or mode \textit{operands}, and extra
information, if required by the specified instructions the JVM were executing at
that time.

The \textit{bytecode} instruction set was designed to be extremely compact. For
that reason, the vast majority of instructions are aligned on the boundaries of
a byte. Also, the total number of opcodes is small enough so that it
considerably reduces the size of files before being loaded by the JVM.

Even though the JVM is a stack machine and a register machine at the same time,
it has no registers that allow it to store arbitrary values, and everything must
be pushed onto the stack before it can be used in a calculation. The JVM was
designed to be hybrid, but at the same time stack-based, in order to facilitate
efficient implementation of (now discontinued) architectures known to be
register-lacking in nature. Therefore, \textit{bytecode} instructions operate
primarily on the stack. 

\newpage
\subsection*{Types}
Just like the Java programming language, the JVM operates on two different kinds
of types: \textit{primitive types} and \textit{reference types}.

The JVM is programmed to expect that nearly all type checking is done before run
time, most normally by a compiler, and it is not the JVM's responsibility to do
it itself. Values of primitive types don't have to be tagged or marked for their
types to be determined at run time. What the JVM does is create different sets
of operations that distinguish the operand
types and behave differently according to those types. For example, even though
the abstract process for the addition operation is the same one for any numeric
value, the JVM has different instructions for the addition operation of each
data type. \parencite{vmachinedesign}

The Java Virtual Machine also has support for objects. An object is one of two
things: a dynamically allocated class instance or an array. A reference to an
object is considered to be of type \textit{reference}\footnote{The
\textit{reference} type is no more than a classic pointer with a different name}.

\subsubsection*{Numeric types}
The numeric types consist of the \textit{integral types} and the
\textit{floating point types}.

The integral types are the following:

    - \textit{\textbf{byte}}, whose values are 8-bit signed two's complement
    integers, and whose default value is zero. The range of accepted values is
    from \(-2^7\) to \(2^7\) inclusive.

    - \textit{\textbf{short}}, whose values are 16-bit signed two's complement
    integers, and whose default value is zero. The range of accepted values is
    from \(-2^{15}\) to \(2^{15}\) inclusive.

    - \textit{\textbf{int}}, whose values are 32-bit signed two's complement
    integers, and whose default value is zero. The range of accepted values is
    from \(-2^{31}\) to \(2^{31}\) inclusive.
    
    - \textit{\textbf{long}}, whose values are 64-bit signed two's complement
    integers, and whose default value is zero. The range of accepted values is
    from \(-2^{63}\) to \(2^{63}\) inclusive.

    - \textit{\textbf{char}}, whose values are 16-bit unsigned integers
    representing \textit{Unicode} code points in the Basic Multilingual Plane,
    encoded with \textit{UTF-16}, and whose default value is the null code point
    (u0000). The range of accepted values is from 0 to 65535 inclusive.

The floating points are:

    - \textbf{\textit{float}}, whose values are elements of the float value set or, where
    supported, the float-extended-exponent value set, and whose default value is
    positive zero.
    
    - \textbf{\textit{double}}, whose values are elements of the double value set or,
    where supported, the double-extended-exponent value set, and whose default
    value is positive zero.

\newpage
\section*{The creation of the executable program}
During the execution of the Java compiler, a \textit{class} file is created.
That file, unfortunately, is far from being ready to be executed in a machine,
which makes the class file totally dependent on the JVM environment. Thanks to
the JVM, the \textit{class} file has an execution environment and an underlying
platform to use as a sandbox when executing the program in question. The
middleman the Java Virtual Machine is, does not only provide the playground for
the \textit{class} file, it is also the intermediary for the exchange process of
services and resources. When breaking down the main processes the JVM carries
out in the early stages of program development, three main ones stand apart.
These are called \emph{loading, linking and initialization}.

\subsection*{Loading}
The process of loading can be defined in many different ways. The
\textit{Official Java Virtual Machine Specification} will be used henceforth,
and any mention or reference related to it will be described as being part of
the \textit{Official Specification}. The aforementioned specification states
that the process of loading is \textit{``the process of finding the binary
representation of a class or interface type with a particular name and creating
a class or interface from the binary representation''}.

There are two main types of class loaders that JVM provides: (1) the
\textit{bootstrap class loader} and (2) the \textit{user-defined class loader}.
The bootstrap class loader is the default class loader, strictly defined and
specified in the JVM, and loads class files accordingly to the specification. On
the other hand, the user-defined class loader is designed to be able to
implement vendor-specific implementations and can load classes in a custom way
via the \textit{java.lang.Class} instance.

Normally, a class loader stores the binary representation of the type at load
time, in advance, or in relation to a specific group of classes. If any problem
is encountered in the early stages of the loading process, the JVM waits until
that specific class is called upon to report the error, and only reports it if
referenced, even if the error remains.

Therefore, the loading process can be boiled down to these main functions: (1)
create a binary stream of data from the class file and (2) parse the binary data
according to the internal data structure.\footnote{Even though the JVM is able
to carry out much more in the linking phase, these two processes are considered
the most important ones.}

\subsection*{Linking}
In the same fashion as the last section, according to the \textit{Official
Specification}, the process of linking is the \textit{``process of taking a
class or interface and combining it into the run-time state of the JVM so that
it can be executed''}.

This linking process starts with a phase called \textit{verification} of the
class. Here is where the JVM makes sure that the code follows the semantic rules
of the language and its addition does not create any disruption in the JVM
environment. Even though this process is well defined by the standard
implementation of the JVM, the specification is intentionally flexible enough
for vendor-specific JVM implementers to decide when the linking activities take
place of the specific processes to follow.

Many little processes happen during the linking phase. First, there is a list of
exceptions specified by the JVM to throw under specific circumstances. The JVM
completes these checks right from the beginning, to make sure parsed binary data
into the internal data structure does not make the program crash. Also, checking
is done to make sure the structure of the binary data aligns with the format it
expects. Even though multiple verifications take place at multiple stages, it is
generally considered that the official verification begins with the linking
process.

Once the verification is complete, JVM allocates the memory for the class
variables and initialies them according to their respective types. The actual
initialization (with user-defined values) does not occur until the next
initialization phase. This process has the name of \textit{preparation}.

Finally\footnote{And optionally.}, in the \textit{resolution} phase, JVM locates
the references of any classes, fields, methods, etc.\ in the \textit{constant
pool}\footnote{Also called symbol table.} and determines their real value. Just
like many features of the JVM, the Java Symbolic reference resolution is open to
vendor-specific implementation. To put it in layman's terms, the verification
process checks that the binary representation of a class has the correct
structure.

Therefore, the linking process involves the following functions: (1)
verification, (2) preparation, and (3) resolution.

\subsection*{Initialization}
As per the \textit{Official Specification}, the \textit{``initialization of a
class or interface consists of executing its class or interface initialization
method''}.

After the class goes through all the previous processes and stages, the
initialization phase makes the class ready for its real use. The proces starts
with the initialization of the class variables with the expected starting value.
Initialization means that the class variables are initialized via some
initialization routine described by the programmer and initialize the class's
direct superclass if it has not already been initialized. The simple
initialization process could be boiled down to two main processes: (1)
initialize class variables with a programmer-specific routine and (2) initialize
super classes if not already initialized.

\section*{Advantages and disadvantages}
As mentioned in the first part of this paper, the use of a
\textit{virtual-machine} based compilation and execution method has both its
advantages and its drawbacks. While many consider the benefits far outweight the
disadvantages, there are still those that believe this scheme to be a misguided one.
In this section, the most prominent advantages and disadvantages of the Java
Virtual Machine can be found. \parencite{jvmspec}

\subsection*{Platform Independent}
One of the biggest pros of the Java Virtual Machine is its platform
independence. The JVM has the ability to be installed on any operating system,
such as Windows, Linux, etc.\ It will transform the programs to bytecode
regardless of the hardware of Operating System to be executed. Nowadays, the JVM
can run on pretty much any modern operating system, and a wide variety of
platforms, ranging from embedded systems to mobile phones.

According to the \textit{Official Java Website} \parencite{gojava} itself, there are between 5 and
10 billion Java devices in the world. This would mean that there may very well
be more devices running Java than people there are on planet Earth. The reader
can easily see that platform independence is one of the biggest factors
influencing such widespread use.

\subsection*{Security}
The Java Virtual Machine is designed in such a way to provide security to the
host computer in terms of their data and program. The JVM performs verification
of the bytecode before running it to prevent the program from performing unsafe
operations such as branching to incorrect locations, which may contain data
rather than instructions. It also allows the JVM to enforce runtime constraints
such as array bounds checking. This means that Java Programs are significanly
less likely to suffer from memory safety flaws such as buffer overflow than
programs written in languages such as C which do not provide such memory safety
guarantees.

The platform does not allow programs to perform certain unsafe operations such
as pointer arithmetic or unchecked type casts. It also does not allow manual
control over memory allocation and deallocation; users are required to rely on
the automatic garbage collector provided by the platform. This also contributes
to type safety and memory safety.

\subsection*{Speed}
Given the fact that Java code is not executed as native code and needs to be run
on a virtual machine for it to work, Java programs will usually take longer to
execute in comparison to C programs, for example.

In the early days of Java, there were many criticisms of its performance. Java
has been demonstrated to run at a speed comparable with optimised native code,
and modern JVM implementations are regularly benchmarked as one of the fastest
language platforms available.

Java bytecode can either be interpreted at run time by the virtual machine, or
it can be compiled at load time or runtime into native code which runs directly
on the computer's hardware. Interpretation is slower than native execution, and
compilation at load time or runtime has an initial performance penalty for the
compilation. Modern performant JVM implementations all use the compilation
approach, so after the initial startup time the performance is similar to native
code.

\subsection*{Cannot experience platform specific features}
The Java Virtual Machine has been developed to be compatible with multiple
Operating Systems. These systems have specific and special features and that
cannot really be changed by a program running on them e.g. the JVM. Therefore,
even if the JVM may be a sandbox for Java Code, it still has its limitations
when it comes to the Operating System itself. A clear example can be that of
Graphical User Interfaces. Even though the JVM may specify exactly the look and
feel of the program, the operating system can easily override that specification
and enforce its own rules.

\subsection*{Correctness}
A program that performs, as expected, it said to be correct. Since a Java
program relies on the Java Virtual Machine to execute it, the JVM must be free
of errors for the program to operate correctly. This reliance on the Java
Virtual Machine introduces a possible point of failure for the program. Luckily,
the Java Virtual Machine software is produced with very high standards, and
therefore it isn't likely to ship with any errors. Regardless, a failure in the
Java Virtual Machine is a possibility to be considered.

\section*{Conclusions}

The Java Virtual Machine has been an invaluable tool and integral part of the
Java language framework. Thanks to its different very well designed features it
receives the worldspread use and attention it does nowadays. Even though many
detractors of the JVM claim its flaws far outweigh its benefits, it is clear
that the global community still uses the language and is content enough with it
every day.

\newpage
\printbibliography

\end{document}
