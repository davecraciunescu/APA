\documentclass{article}

\usepackage{graphicx}

\title{Cloud Computing - Individual}
\author{Dave E. Craciunescu}
\date{2019.05.24}

\begin{document}

\maketitle

\section*{Módulo de Seguridad AWS}

AWS o (Amazon Web Services) es el gran competidor de Microsoft Azure. Dentro de
este apartado analizaremos los servicios análogos a aquellos que Microsoft Azure
nos ofrecía. Cabe mencionar que dentro del módulo de Educación que AWS ha
proporcionado a los diferentes estudiantes de la clase de Ampliación Avanzada de
la UAH, estos no han dispuesto de prácticamente ningún tipo de autorización a la
hora de realizar acciones o emplear servicios, por tanto las herramientas no han
sido probadas de primera mano. Cualquier explicación que se encuentre más abajo
se basa puramente en la documentación oficial de los mismos servicios.

\subsection*{Amazon Cognito}
Amazon Cognito es una herramienta de Amazon Web Services que permite la creación
de aplicaciones cada vez más seguras. El servicio proporciona \textit{pools} de
usuarios en los que la autenticación de estos mismos se realiza automáticamente
mediante Amazon Cognito. De esta manera, no hay que reisntalar la arquitectura y
servicios de autenticación cada vez que se quiera implementar un login seguro
para una aplicación o cualquie tipo de herramienta que implique la
identificación real de los usuarios. 

Gracias a Amazon Cognito, se puede automatizar todo el servicio de
autenticación y hacer tanto el guardado de los hashes de las contraseñas, como
el proceso mismo de introducción de credenciales de manera segura extremadamente
simple. Además, se puede customizar la interfaz gráfica de usuario para que
estos mismos tengan una experiencia mucho más común y placentera. 

En cuanto a este servicio respecta, su paralelo en Azure se encontraría dentro
del servicio de autenticación de Microsoft mismo, que se emplea extensivamente
en Azure Active Directory. En AWS, este mismo se puede exteriorizar y
automatizar de manera muy simple, además, el coste \textit{real} de la
herramienta es considerablemente más adecuado. Por estos mismos motivos se
considera que este servicio sea el superior en cuanto a las dos nubes.

\subsection*{AWS Secrets Manager}
En Microsoft Azure comentamos la implementación, el uso, y las características
de Key Vault. El servicio análogo a este en AWS se llama \textit{AWS Secret
Manager}. Este tiene prácticamente las mismas características que el mencionado
con anterioridad, dado que son, a efectos prácticos, la misma herramienta. 

AWS Secret Manager ayuda a los usuarios a proteger secretos o información
privilegiada importante para las aplicaciones, servicios y recursos IT. El
servicio permite una muy simple manera de rotar y administrar credenciales de
bases de datos, por ejemplo, en base a sus APIs altamente documentadas y bien
diseñadas.

Una característica extremadamente importante de AWS que no se ha encontrado en
Azure Key Vault es la autogeneración del código necesario para la obtención de
una clave en concreto en múltiples lenguajes. Esto es un gran punto a favor para
AWS, dado que la herramienta no va a ser siempre usada en los frameworks
estándar en los que se encuentra.

Con esto comentado, aunque las características sean parecidas y AWS tenga la
autogeneración de código para diferentes lenguajes. Se considera el servicio de
Azure Key Vault superior por la gran simplicidad que este conlleva y el hecho de
que el mismo está integrado con el gran abanico de aplicaciones que es el
framework de Microsoft.

\subsection*{AWS Security Hub}
El Security Hub de Amazon es la versión de AWS del Security Center. Esta
herramienta proporciona una vista simple y fácil de comprender de las
prioridades de alto nivel y las alertas de seguridad de los diferentes servicios
de AWS. En ella se encuentra un array enorme de herramientas de seguridad para
el usuario, desde firewalls y protección de punto a punto a escáner de
cumplimiento y vulnerabilidades. Se podría decir que la herramienta es
prácticamente una navaja suiza de la seguridad empresarial.

Además, en la misma se encuentra un gran sistema de notificaciones y resúmenes
de actividad diarios, junto con la creación automática de informes sobre
vulnerabilidades muy detallados y con información pertinente. Todo esto se
presenta en un dashboard fácil de entender y desde el cual no solo se puede
visualizar la información sino también accionar sobre ella. 

Según la mayoría de las métricas, el servicio de AWS Security Hub parece mucho
más avanzado que su rival de Azure. Desde la integración con Amazon GuardDuty,
Amazon Inspector y Amazon Macie hasta el hecho de la autogeneración de
diferentes informes basados en características definidas por el propio usuario o
administrador de seguridad del sistema. Sin dudarlo, el manager de seguridad de
Amazon es mucho más detallado y avanzado en su funcionalidad.

\subsection*{AWS Shield}
En último lugar, se encuentra Amazon Shield. Este servicio es una herramienta de
protección contra ataque DDoS que salvaguarda aplicaciones que se están
ejecutando en el AWS. De la misma manera que su rivan el Azure, AWS Shield
proporciona mitigación contra ataques automática que minimizan el
\textit{downtime} de los servicios y la latencia que experimentan los usuarios. 

Al analizar las características de las dos herramientas estas son tan similares
que no sería descabellado decir que las compañías copiaron ideas las unas de las
otras. Desde los informes de ataques y en análisis en tiempo real, los dos
niveles de protección, el hecho de que no haga falta ningún tipo de intervención
humana, etc.

Una característica que sí se encuentra en AWS Shield y que Azure no parece tener
es la capacidad de crear layers por encima de los definidos por el propio
software. Con AWS Shield \textit{Advance} el administrador es capaz de escoger
los recursos \textit{específicamente} para proteger la infraestructura. Con esto
se pueden crear reglas de protección altamente sofisticadas que mitiguen hasta
el mejor diseñado ataque DDoS. Además, estas reglas pueden empezar a ser
ejecutadas de manera inmediata, lo que hace el \textit{blue-teaming} mucho más
fácil y la protección de los servicios un juego de niños.

\end{document}
