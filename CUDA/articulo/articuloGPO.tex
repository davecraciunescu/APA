\documentclass{article}

\author{Pablo Acereda}
\title{Artículo sobre Optimizacion mediante Gradiente y Poblaciones}

\begin{document}

\maketitle

\section{Introduction}

El mayor problema a la hora de realizar la optimización de un sistema, ya sea
desde la superficie y forma de una antena o circuitos de microondas, ha sido 
debido a la creciente escala y tamaño de los mismos. Siendo el principal
problema la alta dimensionalidad y la falta de linearidad en los supuestos
presentados como materia de optimización.

Con el propósito de mejorar los procesos que llevan a resultados mejores,
minimizando o maximizando funciones de coste, se han empleado numerosos
algorítmos (como el uso de gradientes, poblaciones evolutivas, etc.); incluso 
la hibridación de varios de estos algoritmos. Dentro de los numerosos 
algoritmos de reducción de costes se puede encontrar, entre los más usados, 
la optimización mediante gradientes (aunque no son del todo óptimos a la 
hora de enfrentar espacios con funciones de coste no lineales).


A causa del auge desenfrenado de los datos a tratar, este algoritmo se ve
afectado de manera negativa; por lo que en este paper se propone el GPO
(Gradient Population Optimization) como algoritmo eficiente que resuelve el
dilema planteado, y que se acrecentará en los años venideros.


Se emplea como arquitectura de ejecución una arquitectura del tipo 
no-von-Neumann y empleado el framework TensorFlow (TF). El algoritmo puede ser
implementado en terminales que cuenten con CPU multi-core, GPUs, FPGA
(field-programmable gate array o matriz de puertas programables en castellano) o
incluso unidades de procesamiento cuántico.

La implementación básica surge de un mecanismo híbrido de optimización local
(cognitivo) y global (social), encontrado en el paradigma de búsqueda aleatoria 
multi-agente (population). Cabe decir que, se ha sustituído el operados social
por un operador gradiente; y una interacción entre los operadores locales y
globales.
\end{document}

